\documentclass[a4paper, 11pt]{article}
    \usepackage{comment} % enables the use of multi-line comments (\ifx \fi) 
    \usepackage{lipsum} %This package just generates Lorem Ipsum filler text. 
    \usepackage{fullpage} % changes the margin
    \usepackage{CJKutf8}
    \usepackage{enumitem}
    \usepackage{titlesec}
    \usepackage{graphicx}     % for figure
    \usepackage{subcaption}   % for figure
    \usepackage[export]{adjustbox}
    \usepackage[most]{tcolorbox}
    \usepackage{xcolor}
    \usepackage{multicol}
    \usepackage{array}
    \usepackage{wrapfig}
    \usepackage{multirow}
    \usepackage{tabu}
    \usepackage{hyperref}
% \renewcommand{\theenumi}{(\alph*)}
  \usepackage{mathtools}

\DeclarePairedDelimiter\ceil{\lceil}{\rceil}
\DeclarePairedDelimiter\floor{\lfloor}{\rfloor}

\titlespacing*{\section}
  {0pt}{0.5\baselineskip}{1\baselineskip}

\titlespacing*{\subsection}
  {0pt}{0.1\baselineskip}{0.1\baselineskip}

\begin{document}
%Header-Make sure you update this information!!!!
\noindent
\begin{center}
  \large\textbf{2018 Fall Data Compression Homework \#2} \\
\end{center}
\begin{CJK}{UTF8}{bsmi}
\normalsize EE 248583 \hfill \textbf{106368002 張昌祺 Justin, Chang-Qi Zhang} \\
Advisor: 電子所 高立人 副教授 \hfill justin840727@gmail.com \\
\null\hfill Date: \today \\
\end{CJK}

\section*{Problem 1}
Let X be a discrete random variable that represents the score of students in the course
“Data Compression”. The distribution of X is given below. Suppose we are using the
Lloyd-Max algorithm, i.e., Nearest Neighbor Condition and Centroid Condition, to find a
4-point scalar quantizer of X. That is, we are trying to find out the four quantization
regions, $R_{1}[x_0, x_1),~R_{2}[x_1, x_2),\\~R_{3}[x_2, x_3)~R_{4}[x_3, x_4)$ as well as the 
reconstructed value $y_1,~y_2,~y_3,~y_4$ corresponding to the four quantization regions. 
The initial codebook $C^0$ is given by $C^0=\{50,~60,~70,~80\}$.

\begin{center}
  \begin{tabular}{ |c|c|c|c|c|c|c|c|c|c|c|c|c|c|c| }
  \hline
   X     & 30 & 45 & 50 & 55 & 60 & 65 & 70 & 75 & 80 & 85 & 90 & 95 & 100 & ow. \\
   \hline 
   Count & 2  & 4  & 4  & 4  & 6  & 8  & 12 & 13 & 10 & 9  & 5  & 2  & 1   & 0 \\
   \hline
  \end{tabular}
  \end{center}
  Note: During the scalar quantization process, each quantization region does not contain the
  right bound of the region, (i.e., the right bound or upper bound of a quantization bin
  is an open interval) except the rightmost overload region.

\begin{enumerate}[label=(\alph*)]
  \item Determine the four quantization regions,$R_{1},~R_{2},~R_{3}~R_{4}$ after 
  the first iteration, i.e., what’s the value of $x_{1},~x_{2},~x_{3}~x_{4}$ after the 
  first iteration?
    \subsection*{Ans}
    
  \item Determine the codebook C 1 after the first iteration, i.e., what’s the value of the
  four reconstructed values, $y_{1},~y_{2},~y_{3}~y_{4}$ after the first iteration?
    \subsection*{Ans}

  \item Determine the four quantization regions,$R_{1},~R_{2},~R_{3}~R_{4}$ after 
  the first iteration, i.e., what’s the value of $x_{1},~x_{2},~x_{3}~x_{4}$ after the 
  second iteration?
    \subsection*{Ans}
  
  \item Determine the codebook C 1 after the first iteration, i.e., what’s the value of the
  four reconstructed values, $y_{1},~y_{2},~y_{3}~y_{4}$ after the second iteration?
    \subsection*{Ans}

\end{enumerate}
\newpage
\section*{Problem 2}
Given an 8-bit gray scale image with dimension 512 x 512. Suppose we are using Vector
Quantization for the compression of the image with block size 8x8 (i.e., codeword
dimension is 8x8), and the codebook size 64. During the coding procedure, the image
width and image height are encoded each with 2 bytes and then transmitted as an
overhead so that the decoder can reconstruct the image correctly. Moreover, we
assume that the decoder also has the same codebook as encoder, so that the codebook does
not have to be transmitted to the decoder.
Please determine the number of bits that should be transmitted if fixed length code is used
for the encoding of the codeword index. (17\%)
\subsection*{Ans}

\section*{Problem 3}
For the probability model in Table1 below, find the real valued tag for the sequence 
$a_1~a_1~a_3~a_2~a_3~\\a_1$, (25\%)
\begin{center}
  \begin{tabular}{ |c|c| }
  \hline
   \textbf{Letter} & \textbf{probability}\\
   \hline 
   $a_1$ & 0.2  \\
   \hline
   $a_2$ & 0.3  \\
   \hline
   $a_3$ & 0.5  \\
   \hline
  \end{tabular}
  \end{center}
\subsection*{Ans}

\section*{Problem 4}
Given the frequency counts shown in Table2 \\
\begin{center}
  \begin{tabular}{ |c|c| }
  \hline
   \textbf{Letter} & \textbf{probability}\\
   \hline 
   $a$ & 2  \\
   \hline
   $b$ & 3  \\
   \hline
   $c$ & 5  \\
   \hline
  \end{tabular}
  \end{center}
Hint: You can send the lower bound of the last symbol for the termination of the encoding
process.

\begin{enumerate}[label=(\alph*)]
  \item What is the word length m required for unambiguous integer encoding?
  \subsection*{Ans}
  \item Use integer implementation to find the binary code for the sequence a a c b c a.
  \subsection*{Ans}
  \item Has E 3 mapping ever been used for this source? Justify your answer.
  \subsection*{Ans}
\end{enumerate}

\end{document}
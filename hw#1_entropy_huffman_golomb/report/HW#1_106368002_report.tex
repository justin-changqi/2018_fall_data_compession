\documentclass[a4paper, 11pt]{article}
    \usepackage{comment} % enables the use of multi-line comments (\ifx \fi) 
    \usepackage{lipsum} %This package just generates Lorem Ipsum filler text. 
    \usepackage{fullpage} % changes the margin
    \usepackage{CJKutf8}
    \usepackage{enumitem}
    \usepackage{titlesec}
    \usepackage{graphicx}     % for figure
    \usepackage{subcaption}   % for figure
    \usepackage[export]{adjustbox}
    \usepackage[most]{tcolorbox}
    \usepackage{xcolor}
% \renewcommand{\theenumi}{(\alph*)}

\titlespacing*{\section}
  {0pt}{0.5\baselineskip}{1\baselineskip}

\titlespacing*{\subsection}
  {0pt}{0.1\baselineskip}{0.1\baselineskip}

\begin{document}
%Header-Make sure you update this information!!!!
\noindent
\begin{center}
  \large\textbf{2018 Fall Data Compression Homework \#1} \\
\end{center}
\begin{CJK}{UTF8}{bsmi}
\normalsize EE 248583 \hfill \textbf{106368002 張昌祺 Justin, Chang-Qi Zhang} \\
Advisor: 電子所 高立人 副教授 \hfill justin840727@gmail.com \\
\null\hfill Due Date: November 12 2018 \\
\end{CJK}

\section*{Problem 1 Entropy}
Let X be a random variable with an alphabet $H=\{1, 2, 3, 4, 5\}$. Please determine 
$H(X)$ for the following three cases of probability mass function $p(i)=prob[X=i]$.~(15\%)
\begin{enumerate}[label=(\alph*)]
  \item $P(1)=P(2)=1/2$:
    \subsection*{Ans}
    Firstly we take a look the result images which are generated by my program for both 
    Lena and baboon gray-level resolution from 8 bits to 1 bit.
  \item $P(i)=1/4, for~i = 1, 2, 3,and~p(4) = p(5) = 1/8$:
    \subsection*{Ans}
  \item $P(i)=2^{-i}, for~i = 1, 2, 3, 4,and~p(5) =  1/16$:
    \subsection*{Ans}
\end{enumerate}
\section*{Problem 2 Huffman Code}
Design a Huffman code C for the source in Problem 1. (15\%)
\begin{enumerate}[label=(\alph*)]
  \item Specify your codewords for individual pmf model in Problem 1.
  \subsection*{Ans}
  \item Compute the expected codeword length and compare with the entropy for your codes in (a).
  \subsection*{Ans}
  \item Design a code with minimum codeword length variance for the pmf model in Problem 1.(b)
  \subsection*{Ans}
\end{enumerate}
\section*{Problem 3 Empirical Distribution C++}
Empirical distribution. In the case a probability model is not known, it can be 
estimated from empirical data. Let’s say the alphabet is $H=\{1, 2, 3,...~,m\}$ . 
Given a set of 
observations of length $N$ , the empirical distribution is given by $p=total~number 
~of~symbol$ $1/N,~for~i=1, 2, 3,..., m$. Please determine the empirical distribution 
for \textbf{santaclaus.txt}, which is an ASCII file with only lower-cased English 
letters (i.e., $a\sim z$), space and CR (carriage return), totally 28 symbols. The 
file can be found on the class web site. Compute the entropy. (14\%)
\subsection*{Ans}
\section*{Problem 4 Huffman Code Encode C++}
Write a program that designs a Huffman code for the given distribution in Problem 3. (14%)
\subsection*{Ans}
\section*{Problem 5 Adaptive Huffman Tree}
Let X be a random variable with an alphabet $H$ , i.e., the 26 lower-case letters.
Use adaptive Huffman tree to find the binary code for the sequence \textbf{a a b b a}.
(24\%) \\
You are asked to use the following 5bits fixed-length binary code as the initial codewords for
the 26 letters. That is \\
a: 00000 \\
b: 00001 \\
\vdots \\
z: 11001 \\
\textbf{Note}: Show the Huffman tree during your coding process.
\subsection*{Ans}
\section*{Problem 6 Golomb Encoding and Decoding.}
\begin{enumerate}[label=(\alph*)]
  \item Find the Golomb code of n=21 when m=4
  \item Find the Golomb code of n=14 when m=4
  \item Find the Golomb code of n=21 when m=5
  \item Find the Golomb code of n=14 when m=5
  \item A two-integer sequence is encoded by Golomb code with m=4 to get the bitstream
  11101111000. What’s the decoded two-integer sequence?
  \item A two-integer sequence is encoded by Golomb code with m=5 to get the bitstream
  11101111000 (the same bitstream as that in (e)). What’s the decoded two-integer sequence?\\
  \textbf{Hint}: The unary code for a positive integer q is simply q 1s followed by a 0.
\end{enumerate}
\end{document}